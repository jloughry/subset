% 3456789012345678901234567890123456789012345678901234567890123456789012
%        1         2         3         4         5         6         7
%
% Format: LaTeX
%
% For information about this file, contact the author by email at
% joe.loughry@lmco.com, or by telephone: +44 (0)7984147430 (mobile), or
% +1 303 932 4720 (office).  The time zone is GMT plus 1 hour.
%
% 20080327.0930 (rjl) initial version
% 20080327.2230 (rjl) I keep getting a weird error (`! LaTeX error:
%	environment theindex undefined.') whenever I try including my normal
%	\usepackage{loughry} in the LaTeX letter class.  Weirdly, there is
%	only one mention in all of Google of this error, and it's unhelpful.
%	Not using \usepackage{loughry} when writing letters in LaTeX seems
%	to be the only workaround for this problem.
%

\documentclass[12pt,a4paper]{letter}
% \usepackage{loughry}
\usepackage{url}

% \usepackage{fullpage}
% \raggedbottom
% \raggedright

\signature{Joe Loughry \\ D.~Phil.~PRS \\ Computing Laboratory}

\address{Joe Loughry \\ Oxford University Computing Laboratory \\
	Wolfson Building \\ Parks Road \\ Oxford, OX1 3QD \\ England}

\begin{document}

\begin{letter}{Professor Donald E.~Knuth \\ Room 215 \\ Oxford}

\opening{Dear Professor Knuth:}

Here is a real-world problem; I call it the banker's problem.  My father
asked about it one day; he used to run
the cheque processing department of a bank downtown.  Every day, he would take
in millions of items---cheques and deposit slips---and run them through
high-speed sorting machines.  Occasionally, a few slips of paper got eaten.
At the end of the day, it's not unusual to be a few million
dollars out of balance.
Someone, usually a clerk,
has the job of matching a pile of loose cheques and deposit slips
to a thick green-bar printout of daily transactions.  As a manual process,
it usually takes someone until ten o'clock at night.

The simplest situation, of course, is when one (\textdollar 5,000) deposit slip
matches another \textdollar $5$,000 cheque.  But more often it takes
a few items to make a match; for example, these sets of items:
$$
\overbrace{\{\$127.14, \$100, \$5000, \$312, \$1000, \$100\}}^\mathrm{credits}
\;\mathrm{and}\;
\overbrace{\{(\$539.14), (\$5000), (\$1100)\}}^\mathrm{debits}
$$
yield solutions:
$$
\$127.14 + \$100 + \$312 = (\$539.14)
$$
$$
\$5000 = (\$5000)
$$
$$
\$1000 + \$100 = (\$1100)
$$

The preferred solution is the smallest number of credits that match one debit.

Here is what I came up with to solve my father's problem: to consider candidate
subsets of credits in order by the number of elements in the subset.

All the 1-subsets:\begin{eqnarray*}
	0001 \\
	0010 \\
	0100 \\
	1000
\end{eqnarray*}

All the 2-subsets:\begin{eqnarray*}
	0011 \\
	0101 \\
	0110 \\
	1001 \\
	1010 \\
	1100
\end{eqnarray*}

All the 3-subsets: \begin{eqnarray*}
	0111 \\
	1011 \\
	1101 \\
	1110
\end{eqnarray*}

All the 4-subsets: \begin{eqnarray*}
	1111
\end{eqnarray*}

As you can see, the binary sequence is unlike lexicographic order; in fact,
\emph{lexicographic order just plain won't work} on a set of a few thousand credits,
because the
computer spends all its time fiddling with stupidly large subsets of the
first few elements in the set 
before it even gets around to looking at the last one.
This way, all
the items in the set are seen early on, and the most likely
solutions---the small subsets---are looked at early.

Back in 1998, I couldn't find anything like it in the \emph{Encyclop\ae dia of
Integer Sequences}.  I wrote an unpublished note about it, and a few years later
Jano van Hemert and L.~Schoofs
contacted me with news that it was a solution to a constraint satisfaction problem
of theirs.  Jano offered a much better recursive algorithm than the one I had
written, so we put all our names on the second version
of the paper.  I'm afraid it's still never been published, but it's been cited in a
few places [Feder 2007, Kandaswamy \emph{et al.} 2008, Maez 2007].  You can imagine my
excitement when you began publishing fascicles for Volume 4.  Is this possibly
something you can use?

The enclosed paper has some pretty graphs contrasting the proposed method
with lexicographic and Gray code ordering, and an algorithm for generating
the binary sequence.

\closing{Respectfully,}

{\bf\large References}

Elie Feder and David Garber.  Towards the computation of the convex
hull of a configuration from its corresponding separating matrix, September 2007.
\url{arXiv:0709.2555}.

Gopi Kandaswami, Anirban Mandal, and Daniel A.~Reed. Fault tolerance and recovery
of scientific workflows on computational grids. In \emph{Eighth IEEE International
Symposium on Cluster Computing and the Grid}, pages 777--782, Lyon, France,
May~19--22, 2008.

Dave Maez. Perl distinct and complete anagram generator, December 6th, 2007.
\url{http://dharmadevil.com/code/anagram.pl.html}.

Encl: `Efficiently Enumerating the Subsets of a Set'

\end{letter}

\end{document}

